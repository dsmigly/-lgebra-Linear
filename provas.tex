\documentclass[11pt,a4paper]{article}
\usepackage{estilosexercicios}

% ---------------------------------------------------
\title{Álgebra Linear}
\author{MAT5730}
\date{2 semestre de 2019}

\begin{document}
\maketitle
\tableofcontents
\newpage
\begin{comment}

\begin{center}
\large\textbf{\textcolor{Floresta}{Provas}}\\
\end{center}

\end{comment}

\section{\textcolor{Floresta}{Prova 1}}

\exercicio{1} Sejam $a,b,c,d \in \mathbb{R}.$ Encontre o valor de
\[
\det \begin{bmatrix}
a & b & c & d\\
9 & 8 & 7 & 6\\
1 & 1 & 1 & 1\\
2020 & 2018 & 2017 & 2016
\end{bmatrix}
\]
\solucao{}

\exercicio{2} Seja $V$ um $K$-espaço vetorial e $T \in \mathcal{L}(V).$ Seja $W \subseteq V$ um subespaço $T$-invariante de $V.$
\dividiritens{
\task[\pers{a}] Mostre que, se $T$ é diagonalizável, então a restrição $T \upharpoonleft_W$ é diagonalizável.
\task[\pers{b}] Seja $\mbox{Spec } T = \{ \lambda_1, \ldots, \lambda_n \}$ o conjunto de autovalores de $T,$ onde $\lambda_i \neq \lambda_j$ para $i \neq j.$ Quantos subespaços $T$-invariantes o espaço vetorial $V$ possui?
}
\solucao{}
\exercicio{3} Encontre o polinômio característico e o polinômio minimal da matriz
\[
\begin{bmatrix}
1 & -1 & 1 & -1 & 1 & -1 \\
1 & -1 & 1 & -1 & 1 & -1 \\
1 & -1 & 1 & -1 & 1 & -1 \\
1 & -1 & 1 & -1 & 1 & -1 \\
1 & -1 & 1 & -1 & 1 & -1 \\
1 & -1 & 1 & -1 & 1 & -1
\end{bmatrix} \in \mathcal{M}_6(K)
\]
\solucao{}
\exercicio{4} Dentre as três matrizes abaixos, quais delas são semelhantes?
\[
A = \begin{pmatrix}
1 & 1 & 1 \\
-1& -1 & -1 \\
1 & 1 & 1
\end{pmatrix}  \quad B = \begin{pmatrix}
1 & 0 & 0 \\
0& 0 & 0 \\
0 & 0 &0
\end{pmatrix} \quad C = \begin{pmatrix}
-1 & -2 & -3 \\
2& 4 & 6 \\
-1& -2 &-3
\end{pmatrix}
\]
\solucao{}
\exercicio{5} Seja $V = \mathcal{P}_3(\mathbb{R})$ o espaço vetorial formado por todos os polinômios de grau menor ou igual a $3.$ Considere o operador $T \in \mathcal{L}(V)$ dado por
\[
T(f(x)) = (x-1) f^{\prime}(x)
\]
\dividiritens{
\task[\pers{a}] Encontre o polinômio característico e o polinômio minimal de $T.$
\task[\pers{b}] $T$ é diagonalizável? Em caso afirmativo, apresente uma base de autovetores para $V.$
}
 \end{document}