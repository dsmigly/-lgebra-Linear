\documentclass[11pt,a4paper]{article}
\usepackage{estilosexercicios}

% ---------------------------------------------------
\title{Álgebra Linear}
\author{MAT5730}
\date{2 semestre de 2019}

\begin{document}
\maketitle
\tableofcontents
\newpage
\begin{comment}

\begin{center}
\large\textbf{\textcolor{Floresta}{Lista 1}}\\
\end{center}

\end{comment}

\section{\textcolor{Floresta}{Lista para P3}}

\begin{exercicio}
 Seja $T$ um operador autoadjunto. Mostre que $\Ker T^2 = \Ker T.$
\end{exercicio} 

\solucao{Vamos mostrar as duas inclusões:
\begin{itemize}
    \item $\Ker T \subset \Ker T^2:$ é óbvio, bastando notar que, se $x \in \Ker T,$ então $T(x) = 0,$ e consequentemente \[T^2(x) = T(\textcolor{Red}{T(x)}) = T(\textcolor{Red}{0}) = 0.\]
    
    \item $\Ker T^2 \subset \Ker T:$ Sendo $T$ autoadjunto, então para todos $u, w \in V$ temos que
\[
\prin{T(u), w} = \prin{u, T(w)}
\]
Escrevendo $w = T(v),$ ficamos com
\[
\prin{T(u), \textcolor{Mahogany}{w}} = \prin{u, T(\textcolor{Mahogany}{w})} \Rightarrow \prin{T(u), \textcolor{Mahogany}{T(v)}} = \prin{u, T(\textcolor{Mahogany}{T(v)})} \Rightarrow \prin{T(u), T(v)} = \prin{u, T^2(v)}
\]
Seja $x \in \Ker T^2.$ Então, $T^2(x) = 0.$ Fazendo $u = v = x$ na expressão acima, vem
\[\prin{T(x), T(x)} = \prin{x, \textcolor{PineGreen}{T^2(x)}} = \prin{x, \textcolor{PineGreen}{0}} = 0 \Rightarrow \prin{T(x), T(x)} = 0 \Rightarrow \norm{T(x)}^2 = 0 \Rightarrow T(x) = 0.\]
Portanto, concluímos que $x \in \Ker T.$
\end{itemize}
Logo, $\Ker T^2 = \Ker T$ se $T$ for autoadjunto.
}

\begin{exercicio}
 
Seja $V$ um espaço complexo com produto interno. Prove que se $\prin{T(v), v}$ é real para
qualquer $v \in V,$ então $T$ é autoadjunto.

\end{exercicio}
\solucao{$(\Rightarrow)$ Seja $T$ um operador auto-adjunto. Então, sabemos que $T^{*} = T.$ Para mostrar que certo $z \in \mathbb{C}$ é real, basta verificar que $z = \overline{z}.$ Dado  $v \in V,$ tem-se que
\[
\langle T(v), v \rangle = \langle v, \textcolor{Brown}{T^{*}}(v) \rangle = \langle v, \textcolor{Brown}{T}(v) \rangle = \overline{\langle T(v),v \rangle} \Rightarrow \langle T(v), v \rangle \in \mathbb{R}
\]
$(\Leftarrow)$ Dado $v \in V,$ se $\prin{T(v), v} \in \mathbb{R}$, então temos que $\prin{T(v), v} = \overline{\prin{T(v), v}}.$ Assim, para todo $v\in V$:
\[
\langle T(v), v \rangle = \overline{ \langle T(v), v \rangle } = \overline{ \langle v, T^{*}(v) \rangle } = \langle T^{*}(v), v \rangle \Rightarrow \langle (T - T^{*})(v), v \rangle = 0.
\]
Como $V$ é um espaço vetorial sobre $\mathbb{C},$ então pelo Exercício 19 da Lista 5 podemos concluir que $T - T^{*}=0,$ acarretando $T = T^{*}.$
}

\begin{exercicio}
 Sejam $S$ e $T$ autoadjuntos. Mostre que $ST$ é autoadjunto se e somente se $ST = TS.$
\end{exercicio}
\solucao{Se $S$ e $T$ são autoadjuntas, então $S = S^{*}$ e $T = T^{*}.$ Logo, temos que
\[
ST=(ST)^*\Leftrightarrow ST=T^*S^*\Leftrightarrow ST=TS,
\]
ou seja, $ST$ é autoadjunta se e somente se $ST=TS$.}

\begin{exercicio}
 
Para cada uma das seguintes matrizes simétricas $A,$, encontre uma matriz ortogonal $P$ para a qual a matriz $P^tAP$ seja diagonal, ou seja, encontre a Forma Alternativa de cada matriz:

\dividiritensdiv{3}{
\task[\pers{a}] $A = \begin{pmatrix}
1 & 2 \\
2 & -2
\end{pmatrix}$
\task[\pers{b}] $B = \begin{pmatrix}
5 & 4 \\
4 & -1
\end{pmatrix}$
\task[\pers{c}] $C = \begin{pmatrix}
7 & 3 \\
3 & -1
\end{pmatrix}$
}
\end{exercicio}
\solucao{
\dividiritens{
\task[\pers{a}] Vamos primeiramente encontrar os autovalores e autovetores de $A.$ Temos:
\[
p_A(\lambda) = \det(A - \lambda I) = \det \left( \begin{pmatrix}
1 & 2 \\
2 & -2
\end{pmatrix} - \begin{pmatrix}
\lambda & 0 \\
0 & \lambda
\end{pmatrix}  \right) = \det \left( \begin{pmatrix}
1-\lambda & 2 \\
2 & -2-\lambda
\end{pmatrix}  \right) = \lambda^2 + \lambda - 6
\]
Como $p_A(\lambda) = (\lambda + 3)(\lambda - 2),$ os autovalores são $\lambda_1 = -3$ e $\lambda_2 = 2.$ Encontremos autovetores associados a eles. Seja $v_i = (x_i,y_i)$ um autovetor associado a $\lambda_i, i =1,2.$ Então
\[
Av=  \lambda_1 v_1 \Rightarrow \begin{pmatrix}
1 & 2 \\
2 & -2
\end{pmatrix} \begin{pmatrix}
x_1 \\ y_1
\end{pmatrix} =  \begin{pmatrix}
-3x_1 \\ -3y_1
\end{pmatrix} \Rightarrow \begin{cases}
4x_1+2y_1 = 0 \\
2x_1 + y_1 = 0
\end{cases} \Rightarrow v_1 = y_1 \left(-\frac{1}{2}, 1 \right).
\]
Podemos tomar $v_1 = (-1,2)$ como autovetor associado a $\lambda_1 = -3.$ Analogamente,
\[
Av=  \lambda_2 v_2 \Rightarrow \begin{pmatrix}
1 & 2 \\
2 & -2
\end{pmatrix} \begin{pmatrix}
x_2 \\ y_2
\end{pmatrix} =  \begin{pmatrix}
2x_2 \\ 2y_2
\end{pmatrix} \Rightarrow \begin{cases}
-x_2+2y_2 = 0 \\
2x_2 - 4y_2 = 0
\end{cases} \Rightarrow v_2 = y_2 \left(2, 1 \right).
\]

Logo, temos que $v_2 = (2,1)$ é um autovetor associado a $\lambda_2 = 2.$ 
Podemos escolher as colunas da matriz $P$ como autovetores ortogonais normalizados de $A,$ e os elementos diagonais serão os autovalores associados.
Já que $A$ é simétrica, temos que$\prin{v_1, v_2} = 0.$ Basta agora normalizá-los. Dado que $\norm{v_1} = \norm{v_2} = \sqrt{5},$ os autovetores ortogonais normalizados de $A$ que procuramos são
\[
w_1 = \frac{v_1}{\norm{v_1}} = \frac{(-1,2)}{\sqrt{5}} = \left(-\frac{\sqrt{5}}{5}, \frac{2 \sqrt{5}}{5} \right) \quad \mbox{e} \quad w_2 = \frac{v_2}{\norm{v_2}} = \frac{(2,1)}{\sqrt{5}} = \left(\frac{2\sqrt{5}}{5}, \frac{ \sqrt{5}}{5} \right) 
\]
Portanto, temos que
\[
P = \begin{pmatrix}
-\frac{\sqrt{5}}{5} & \frac{2\sqrt{5}}{5} \\
\frac{2 \sqrt{5}}{5} & \frac{\sqrt{5}}{5} 
\end{pmatrix}
\]
De fato, uma rápida verificação confirma que
\[
 \begin{pmatrix}
-3 & 0 \\
0 & 2
\end{pmatrix} = \begin{pmatrix}
-\frac{\sqrt{5}}{5} & \frac{2\sqrt{5}}{5} \\
\frac{2 \sqrt{5}}{5} & \frac{\sqrt{5}}{5} 
\end{pmatrix}
 \begin{pmatrix}
1 & 2 \\
2 & -2
\end{pmatrix}
\begin{pmatrix}
-\frac{\sqrt{5}}{5} & \frac{2\sqrt{5}}{5} \\
\frac{2 \sqrt{5}}{5} & \frac{\sqrt{5}}{5} 
\end{pmatrix}
\]

\task[\pers{b}] Encontremos os autovalores e autovetores de $B:$
\[
p_B(\lambda) = \det(B - \lambda I) = \det \left( \begin{pmatrix}
5 & 4 \\
4 & -1
\end{pmatrix} - \begin{pmatrix}
\lambda & 0 \\
0 & \lambda
\end{pmatrix}  \right) = \det \left( \begin{pmatrix}
5-\lambda & 4 \\
4 & -1-\lambda
\end{pmatrix}  \right) = \lambda^2 - 4\lambda - 21
\]
Como $p_B(\lambda) = (\lambda + 3)(\lambda - 7),$ os autovalores são $\lambda_1 = -3$ e $\lambda_2 = 7.$ Encontremos autovetores associados a eles. Seja $v_i = (x_i,y_i)$ um autovetor associado a $\lambda_i, i =1,2.$ Então
\[
Bv=  \lambda_1 v_1 \Rightarrow \begin{pmatrix}
5 & 4 \\
4 & -1
\end{pmatrix} \begin{pmatrix}
x_1 \\ y_1
\end{pmatrix} =  \begin{pmatrix}
-3x_1 \\ -3y_1
\end{pmatrix} \Rightarrow \begin{cases}
8x_1+4y_1 = 0 \\
4x_1 + 2y_1 = 0
\end{cases} \Rightarrow v_1 = y_1 \left(-\frac{1}{2}, 1 \right).
\]
Podemos tomar $v_1 = (-1,2)$ como autovetor associado a $\lambda_1 = -3.$ Analogamente,
\[
Bv=  \lambda_2 v_2 \Rightarrow \begin{pmatrix}
5 & 4 \\
4 & -1
\end{pmatrix} \begin{pmatrix}
x_2 \\ y_2
\end{pmatrix} =  \begin{pmatrix}
7x_2 \\ 7y_2
\end{pmatrix} \Rightarrow \begin{cases}
-2x_2+4y_2 = 0 \\
4x_2 - 8y_2 = 0
\end{cases} \Rightarrow v_2 = y_2 \left(2, 1 \right).
\]

Logo, temos que $v_2 = (2,1)$ é um autovetor associado a $\lambda_2 = 7.$ 
Sendo $B$ simétrica, $\prin{v_1, v_2} = 0.$ Basta agora normalizá-los. Dado que $\norm{v_1} = \norm{v_2} = \sqrt{5},$ os autovetores ortogonais normalizados de $B$ que procuramos são
\[
w_1 = \frac{v_1}{\norm{v_1}} = \frac{(-1,2)}{\sqrt{5}} = \left(-\frac{\sqrt{5}}{5}, \frac{2 \sqrt{5}}{5} \right) \quad \mbox{e} \quad w_2 = \frac{v_2}{\norm{v_2}} = \frac{(2,1)}{\sqrt{5}} = \left(\frac{2\sqrt{5}}{5}, \frac{ \sqrt{5}}{5} \right) 
\]
Portanto, temos que
\[
P = \begin{pmatrix}
-\frac{\sqrt{5}}{5} & \frac{2\sqrt{5}}{5} \\
\frac{2 \sqrt{5}}{5} & \frac{\sqrt{5}}{5} 
\end{pmatrix}
\]
Assim,
\[
 \begin{pmatrix}
-3 & 0 \\
0 & 7
\end{pmatrix} = \begin{pmatrix}
-\frac{\sqrt{5}}{5} & \frac{2\sqrt{5}}{5} \\
\frac{2 \sqrt{5}}{5} & \frac{\sqrt{5}}{5} 
\end{pmatrix}
 \begin{pmatrix}
5 & 4 \\
4 & -1
\end{pmatrix}
\begin{pmatrix}
-\frac{\sqrt{5}}{5} & \frac{2\sqrt{5}}{5} \\
\frac{2 \sqrt{5}}{5} & \frac{\sqrt{5}}{5} 
\end{pmatrix}
\]

\task[\pers{c}] Encontremos os autovalores e autovetores de $C:$
\[
p_C(\lambda) = \det(C - \lambda I) = \det \left( \begin{pmatrix}
7 & 3 \\
3 & -1
\end{pmatrix} - \begin{pmatrix}
\lambda & 0 \\
0 & \lambda
\end{pmatrix}  \right) = \det \left( \begin{pmatrix}
7-\lambda & 3 \\
3 & -1-\lambda
\end{pmatrix}  \right) = \lambda^2 - 6\lambda - 16
\]
Como $p_B(\lambda) = (\lambda + 2)(\lambda - 8),$ os autovalores são $\lambda_1 = -2$ e $\lambda_2 = 8.$ Encontremos autovetores associados a eles. Seja $v_i = (x_i,y_i)$ um autovetor associado a $\lambda_i, i =1,2.$ Então
\[
Cv=  \lambda_1 v_1 \Rightarrow \begin{pmatrix}
7 & 3 \\
3 & -1
\end{pmatrix} \begin{pmatrix}
x_1 \\ y_1
\end{pmatrix} =  \begin{pmatrix}
-2x_1 \\ -2y_1
\end{pmatrix} \Rightarrow \begin{cases}
9x_1+3y_1 = 0 \\
3x_1 + y_1 = 0
\end{cases} \Rightarrow v_1 = y_1 \left(-\frac{1}{3}, 1 \right).
\]
Podemos tomar $v_1 = (-1,3)$ como autovetor associado a $\lambda_1 = -2.$ Analogamente,
\[
Cv=  \lambda_2 v_2 \Rightarrow \begin{pmatrix}
7 & 3 \\
3 & -1
\end{pmatrix} \begin{pmatrix}
x_2 \\ y_2
\end{pmatrix} =  \begin{pmatrix}
8x_2 \\ 8y_2
\end{pmatrix} \Rightarrow \begin{cases}
-x_2+3y_2 = 0 \\
3x_2 - 9y_2 = 0
\end{cases} \Rightarrow v_2 = y_2 \left(3, 1 \right).
\]

Logo, temos que $v_2 = (3,1)$ é um autovetor associado a $\lambda_2 = 8.$ 
Sendo $B$ simétrica, $\prin{v_1, v_2} = 0.$ Basta agora normalizá-los. Dado que $\norm{v_1} = \norm{v_2} = \sqrt{10},$ os autovetores ortogonais normalizados de $B$ que procuramos são
\[
w_1 = \frac{v_1}{\norm{v_1}} = \frac{(-1,3)}{\sqrt{10}} = \left(-\frac{\sqrt{10}}{10}, \frac{3 \sqrt{10}}{10} \right) \quad \mbox{e} \quad w_2 = \frac{v_2}{\norm{v_2}} = \frac{(3,1)}{\sqrt{10}} = \left(\frac{3\sqrt{10}}{10}, \frac{ \sqrt{10}}{10} \right) 
\]
Portanto, 
\[
P = \begin{pmatrix}
-\frac{\sqrt{10}}{10} & \frac{3\sqrt{10}}{10} \\
\frac{3 \sqrt{10}}{10} & \frac{\sqrt{10}}{10} 
\end{pmatrix}
\]
Logo,
\[
 \begin{pmatrix}
-2 & 0 \\
0 & 8
\end{pmatrix} = \begin{pmatrix}
-\frac{\sqrt{10}}{10} & \frac{3\sqrt{10}}{10} \\
\frac{3 \sqrt{10}}{10} & \frac{\sqrt{10}}{10} 
\end{pmatrix}
 \begin{pmatrix}
7 & 3 \\
3 & -1
\end{pmatrix}
\begin{pmatrix}
-\frac{\sqrt{10}}{10} & \frac{3\sqrt{10}}{10} \\
\frac{3 \sqrt{10}}{10} & \frac{\sqrt{10}}{10}  
\end{pmatrix}
\]
} 
}


\begin{exercicio}

Seja $T \in \mathcal{L}(\mathbb{C}^2)$ definido por $T(1, 0) = (1 + i, 2)$ e $T(0, 1) = (i, i).$ $T$ é normal?
\end{exercicio}

\solucao{Seja $A \in \mathcal{M}_2(\mathbb{C})$ a matriz que representa $T.$ Pelas informações do enunciado, se
\[
A = \begin{pmatrix}
\alpha & \beta \\
\gamma & \delta
\end{pmatrix},
\]
devemos ter
\[
\begin{pmatrix}
\alpha & \beta \\
\gamma & \delta
\end{pmatrix} \begin{pmatrix}
1 \\ 0
\end{pmatrix} =  \begin{pmatrix}
1+i \\ 2
\end{pmatrix} \quad \mbox{e} \quad \begin{pmatrix}
\alpha & \beta \\
\gamma & \delta
\end{pmatrix} \begin{pmatrix}
0 \\ 1
\end{pmatrix} =  \begin{pmatrix}
i \\ i
\end{pmatrix}.
\]
Daí, segue que $\alpha = 1 + i, \gamma = 2, $ e $\beta = \delta = i.$ Portanto, a matriz de $T$ na base canônica de $\mathbb{C}^2$ é \[\begin{pmatrix}
1+i & i \\
2 & i
\end{pmatrix}.\]
Observe que a matriz adjunta $T^{*}$ corresponde ao hermitiano de $A,$ ou seja,
\[
[T^{*}]_{\mbox{can}} = A^{*} = A^{H} = \overline{A}^t.
\]
Logo, temos que a matriz procurada é
\[[T^{*}]_{\mbox{can}} = A^{*} = \overline{\begin{pmatrix}
1+i & i \\
2 & i
\end{pmatrix}}^t = \begin{pmatrix}
1-i & -i \\
2 & -i
\end{pmatrix}^t = \begin{pmatrix}
1-i & 2\\
-i & -i
\end{pmatrix}\]
Observe que
\[AA^{*} = \begin{pmatrix}
1+i & i \\
2 & i
\end{pmatrix}\begin{pmatrix}
1-i & 2\\
-i & -i
\end{pmatrix} = \begin{pmatrix}
3 & 3 +2i\\
3-2i & 5
\end{pmatrix}
\]
\[
A^{*}A = 
\begin{pmatrix}
1-i & 2\\
-i & -i
\end{pmatrix} \begin{pmatrix}
1+i & i \\
2 & i
\end{pmatrix} =
\begin{pmatrix}
6 & 1+3i\\
1-3i & 2
\end{pmatrix}
\]
Logo $TT^{*} \neq T^{*}T$ nesse caso, ou seja, $T$ não é um operador normal.
}

\begin{exercicio}
 Sejam $V$ um espaço unitário e $T \in \mathcal{L}(V)$ um operador normal. Mostre que
\dividiritens{
\task[\pers{a}] $T$ é autoadjunto se e somente se $\mbox{Spec }T \subseteq \mathbb{R}.$
\task[\pers{b}] $T$ é unitário se e somente se, para todo $\lambda \in \mbox{Spec }T, \abs{\lambda} = 1.$
}
\end{exercicio}
\solucao{
\dividiritens{
\task[\pers{a}] Seja $\lambda$ um autovalor de $T.$ Considere $v \neq 0$ um autovetor associado a $\lambda,$ ou seja, tal que $T(v) = \lambda v.$ Observe que
\[
\begin{array}{rcl}
\lambda \prin{v,v} &=& \prin{\textcolor{Cerulean}{\lambda v},v } \\
&=& \prin{\textcolor{Cerulean}{T(v)},v } \\
&=& \prin{v, \textcolor{Blue}{T^{*}}(v)} \\
&=& \prin{v, \textcolor{Blue}{T}(v)} \\
&=& \overline{\prin{\textcolor{Cerulean}{T(v)},v}} \\
&=& \overline{\prin{\textcolor{Cerulean}{\lambda v},v}} \\
&=& \overline{\lambda} \prin{v,v}
\end{array}
\]
Como $v \neq 0,$ então $\prin{v,v} \neq 0.$ Portanto,
\[
\lambda \prin{v,v} = \overline{\lambda} \prin{v,v} \Rightarrow \lambda = \overline{\lambda} \Rightarrow \lambda \in \mathbb{R}.
\]
A recíproca é análoga.
\task[\pers{b}] Sabemos que um operador unitário preserva o produto interno, isto é, para todos $u, v \in V,$
\[
\prin{T(u), T(v) } = \prin{u,v}
\]
tome $\lambda$ um autovalor de $T$ e  $w \neq 0$ um autovetor associado a $\lambda.$ Então, $T(w) = \lambda w,$ e daí
\[
\prin{w,w} = \prin{T(w), T(w) } = \prin{\lambda w, \lambda w} = \lambda \overline{\lambda} \prin{w,w} = \abs{\lambda}^2 \prin{w,w}
\]
Sendo $\prin{w,w} \neq 0,$ pois $w$ é um autovetor, então concluímos que $\abs{\lambda}^2 = 1 \Rightarrow \abs{\lambda} = 1.$
Novamente, a recíproca é análoga.
}
}

\begin{exercicio}\textcolor{red}{incompleto}
Seja $V = \mathbb{R}^n$ com produto interno usual e $T \in \mathcal{L}(\mathbb{R}^n).$ Suponhamos que $v_1 = (1, 1, \ldots , 1), v_2 = (1, 1, \ldots , 1, 0), \ldots , v_n = (1, 0, .\ldots, 0)$ sejam autovetores de $T.$
Mostre que $T$ é autoadjunto se e somente se $T$ possui um único autovalor.
\end{exercicio}
\solucao{$(\Rightarrow)$ Suponha que $T$ seja autoadjunto. Considere $\lambda_i \in \mathbb{R}$ (pelo exercício 40) autovalores distintos de $T.$ Temos portanto que:
\[
\begin{array}{ccc}
T(1, 1, \ldots , 1, 1, 1) &=& \lambda_1(1, 1, \ldots , 1, 1, 1) \\
T(1, 1, \ldots , 1, 1, 0) &=& \lambda_2(1, 1, \ldots , 1, 1, 0) \\
T(1, 1, \ldots , 1, 0, 0) &=& \lambda_3(1, 1, \ldots , 1, 0, 0)  \\
& \vdots & \\
T(1, 0, \ldots , 0, 0, 0) &=& \lambda_n(1, 0, \ldots , 0, 0, 0) \\
\end{array}
\]
Assim, veja que
\[
T(2, 1, 0, \ldots, 0) = 2T(v_{n}) + T(v_{n-1}) = (2
\]

}

\begin{exercicio}
Consideremos $V = \mathbb{C}$ como um $\mathbb{R}$-espaço vetorial.
\dividiritens{
\task[\pers{a}] Mostre que $\prin{\alpha, \beta} = \mbox{Re}(\alpha \overline{\beta})$ é um produto interno em $\mathbb{C}.$

\task[\pers{b}] Para cada $\gamma \in V,$ seja a função $M_{\gamma} \colon \mathbb{C} \to \mathbb{C}$ dada por $M_{\gamma}(\alpha) = \gamma \alpha.$ Mostrar
que $(M_{\gamma})^{*} = M_{\overline{\gamma}}.$

\task[\pers{c}] Para quais $\gamma \in \mathbb{C}$, $M_{\gamma}$ é autoadjunto?

\task[\pers{d}] Para quais $\gamma \in \mathbb{C}$, $M_{\gamma}$ é unitário?
}
\end{exercicio}
\solucao{
\dividiritens{
\task[\pers{a}] Vamos verificar os axiomas de produto interno:
\begin{itemize}
    \item[$\clubsuit$] $\prin{\alpha+\beta, \gamma} = \prin{\alpha, \gamma} + \prin{\beta, \gamma}.$
    Sejam $\alpha = \alpha_1 + i \alpha_2, \beta = \beta_1 + \beta_2 i, \gamma = \gamma_1 + \gamma_2 i \in \mathbb{C}.$ Então
    \[
    \prin{\alpha+\beta, \gamma} =  \mbox{Re}((\alpha+\beta) \overline{\gamma}) =  \mbox{Re}((\alpha_1 + i \alpha_2+\beta_1 + \beta_2 i) \overline{ \gamma_1 + \gamma_2 i}) = \]\[ \mbox{Re}(((\alpha_1 + \beta_1) + i (\alpha_2+\beta_2))(\gamma_1 - \gamma_2 i)) = (\alpha_1 + \beta_1)\gamma_1 + (\alpha_2+\beta_2)\gamma_2 = 
    \]
    \item[$\textcolor{Red}{\varheart}$]
    \item[$\spadesuit$]
    \item[$\textcolor{Red}{\vardiamond}$]
\end{itemize}
}

}
\begin{exercicio} Para cada uma das seguintes matrizes $A,$ encontre uma matriz ortogonal $T$ tal que a matriz $T^tAT$ seja diagonal.
\dividiritens{
\task[\pers{a}] $A = \begin{bmatrix}
11 & 2 & -8 \\
2 & 2 & 10 \\
-8 & 10 & 5
\end{bmatrix}$
\task[\pers{b}] $A = \begin{bmatrix}
5 & -1 & -1 \\
-1 & 5 & -1 \\
-1 & -1 & 5
\end{bmatrix}$
\task[\pers{c}] $A = \begin{bmatrix}
0 & 0 & 1 \\
0 & 1 & 0 \\
1 & 0 & 0
\end{bmatrix}$
}
\end{exercicio}
\solucao{
\dividiritens{
\task[\pers{a}]
\task[\pers{b}]
\task[\pers{c}] Vamos encontrar os autovalores de $A.$ Observe que o polinômio característico de $A$ será $p_A(\lambda) = \lambda^3 - \lambda^2 - \lambda + 1 = (\lambda - 1)^2(\lambda + 1).$ Logo, os autovalores são $\lambda_1 = 1$ e $\lambda_2 = -1.$ Encontremos os respectivos autovetores associados:
\begin{itemize}
    \item Para $\lambda_1 = 1$ e $v = (x,y,z),$ temos
    \[
    (A - I)v = 0 \Rightarrow \begin{bmatrix}
-1 & 0 & 1 \\
0 & 0 & 0 \\
1 & 0 & -1
\end{bmatrix}\begin{bmatrix}
x \\
y \\
z
\end{bmatrix} = \begin{bmatrix}
0\\
0 \\
0
\end{bmatrix} \Rightarrow \begin{cases}
-x+z = 0 \\
x - z = 0
\end{cases} \Rightarrow v = x(1,0,1) + y(0,1,0)
    \]
    Logo, temos que $\Ker(A - I) = \langle (1,0,1), (0,1,0) \rangle,$ e $(1,0,1)$ e $(0,1,0)$ são autovetores associados a $\lambda_1 = 1.$
      \item Para $\lambda_2 = -1$ e $v = (x,y,z),$ temos
    \[
    (A + I)v = 0 \Rightarrow \begin{bmatrix}
1 & 0 & 1 \\
0 & 2 & 0 \\
1 & 0 & 1
\end{bmatrix}\begin{bmatrix}
x \\
y \\
z
\end{bmatrix} = \begin{bmatrix}
0\\
0 \\
0
\end{bmatrix} \Rightarrow \begin{cases}
x+z = 0 \\
2y = 0 \\
x+z=0
\end{cases} \Rightarrow v = x(1,0,-1)
    \]
    Logo, temos que $\Ker(A + I) = \langle (1,0,-1) \rangle,$ e $(1,0,-1)$é um autovetor associado a $\lambda_2 = -1.$
\end{itemize}
Logo, a matriz $T$ procurada é 
\[
T = \begin{bmatrix}
1 & 0 & 1 \\
0 & 1 & 0 \\
1 & 0 & -1
\end{bmatrix}
\]
Assim, segue que
\[
D = T^tAT \Rightarrow \begin{bmatrix}
1 & 0 & 0 \\
0 &  1 & 0\\
0 & 0 & -1
\end{bmatrix} = \begin{bmatrix}
1 & 0 & 1 \\
0 & 1 & 0 \\
1 & 0 & -1
\end{bmatrix}\begin{bmatrix}
0 & 0 & 1 \\
0 & 1 & 0 \\
1 & 0 & 0
\end{bmatrix}\begin{bmatrix}
1 & 0 & 1 \\
0 & 1 & 0 \\
1 & 0 & -1
\end{bmatrix}
\]
Observe que $T$ é ortogonal, pois $A$ é simétrica.
}
}


\begin{exercicio}\textcolor{red}{conta grande}
Para cada uma das seguintes matrizes $A,$ mostre que ela é uma matriz normal e ache uma matriz unitária $P$ tal que $\overline{P^t}AP$ seja diagonal.
\dividiritensdiv{2}{
\task[\pers{a}] $\begin{bmatrix}
1 & 1 \\
i & 3+2i
\end{bmatrix}$
\task[\pers{b}] $\begin{bmatrix}
2 & i \\
i & 2
\end{bmatrix}$
}
\end{exercicio}
\solucao{
\dividiritens{
\task[\pers{a}]
}
}

\begin{exercicio}
Mostre que se $T$ é normal, então $\Ker T = \Ker T^{*}$ e $\mbox{Im } T = \mbox{Im } T^{*}.$
\end{exercicio}
\solucao{Lembrando que, como $T$ é normal, então $TT^{*} = T^{*}T:$
\begin{itemize}
    \item $\Ker T = \Ker T^{*}:$ Seja $v \in \Ker T.$ Então, $T(v) = 0.$ Logo,
    \[
    T(T^{*}(v)) = T^{*}(\textcolor{Green}{T(v)}) = T^{*}(\textcolor{Green}{0}) = 0 \Rightarrow T^{*}(v) = 0.
    \]
    Assim, $v \in \Ker T^{*}.$ Assim, $\Ker T \subset \Ker T^{*}.$ Analogamente, verifica-se que $\Ker T^{*} \subset \Ker T.$
    \item $\mbox{Im } T = \mbox{Im } T^{*}:$ Mostremos equivalentemente que $\mbox{Im }(TT^{*}) = \mbox{Im }(T).$ Se $TT^{*}(x) = 0,$ então
    \[
    \prin{TT^{*}(x),x} = \prin{T^{*}T(x),x} = \prin{T(x),T(x)} = 0 \Rightarrow T(x) = 0
    \]
    Logo $\Ker(TT^{*}) \subset \Ker(T),$ e temos que $T(x) = 0$ implica $T^{*}(T(x)) = T(T^{*}(x)) = 0,$ de onde segue que $\Ker(T) \subset \Ker(TT^{*})$ e assim $\Ker(T) = \Ker(TT^{*}).$ Logo, $\dim(\mbox{Im }(T)) = \dim(\mbox{Im }(TT^{*})).$ Assim, concluímos que $\mbox{Im } T = \mbox{Im } T^{*}.$
    
\end{itemize}
}

\begin{exercicio}
Mostre que $T \in \mathcal{L}(V)$ é normal se, e somente se, $\norm{T(v)} = \norm{T^{*}(v)} \ \forall v \in V.$
\end{exercicio}
\solucao{Mostremos que $\prin{T(v), T(v)} = \prin{T^{*}(v), T^{*}(v)},$ para todo $v \in V.$ De fato,
\[
\prin{T(v), T(v)} = \prin{v, T^{*}T(v)} = \prin{v, TT^{*}(v)} = \prin{T^{*}(v), T^{*}(v)}.
\]
Como $\norm{T(v)} = \sqrt{\prin{T(v), T(v)}}$ e $\norm{T^{*}(v)} = \sqrt{\prin{T^{*}(v), T^{*}(v)}},$ segue o resultado desejado. 
}

\begin{exercicio}
Mostre que uma matriz triangular é normal se, e somente se, ela é diagonal.
\end{exercicio}
\solucao{}
\begin{exercicio}
Um operador $S$ chama-se anti-autoadjunto se $S^{*} = -S.$ Mostre que um operador $T$ é autoadjunto se, e somente se, o operador $iT$ é anti-autoadjunto.
\end{exercicio}
\solucao{$(\Rightarrow)$ Se $T$ é autoadjunto, então $T^{*} = T.$ Logo,
\[
(iT)^{*} = \overline{i}T^{*} = iT^{*} = iT
\]
Portanto, $iT$ é anti-autoadjunto.
$(\Leftarrow)$ Se $iT$ é anti-autoadjunto, então
\[
iT = (iT)^{*} = \overline{i}T^{*} = iT^{*} 
\]
Portanto, $T^{*} = T$ e $T$ é autoadjunto.
} 

\begin{exercicio}
Seja $V$ um espaço euclidiano e $T \in \mathcal{L}(V)$ é um operador antiautoadjunto. Prove que se $\dim_{\mathbb{R}}(V)$ é ímpar então $T$ é degenerado.
\end{exercicio}
\solucao{}

\begin{exercicio}
Seja $V$ um espaço euclidiano e $T \in \mathcal{L}(V).$ Prove que $T$ é antiautoadjunto se, e somente se, $\prin{T(v), v} = 0 \ \forall v \in V.$
\end{exercicio}
\solucao{}

\begin{exercicio}
Um operador auto-adjunto $T \in \mathcal{L}(V)$ chama-se \emph{positivo definido} (\emph{positivo semidefinido}) se $\prin{T(v), v} > 0$ (equivalentemente, $\prin{T(v), v} \ge 0$) para qualquer vetor $v \in V.$ De maneira semelhante, uma matriz $A \in \mathcal{M}_n(K)$ chama-se \emph{positiva definida} (\emph{positiva semidefinida}) se o operador $A \in \mathcal{L}(K^n)$ cuja matriz na base canônica de $K^n$
é A, é positiva (não negativo).
\textcolor{Red}{NAÕ EXISTE PERGUNTA! CRIAR DEPOIS}
\end{exercicio}
\solucao{}
\begin{exercicio} Determine quais das seguintes matrizes são positivo definidas (semidefinidas):
\dividiritens{
\task[\pers{a}] $\begin{pmatrix}
1 & 1 \\
1 & 1
\end{pmatrix}$
\task[\pers{b}] $\begin{pmatrix}
0 & i \\
-i & 0
\end{pmatrix}$
\task[\pers{c}] $\begin{pmatrix}
0 & 1 \\
-1 & 0
\end{pmatrix}$
\task[\pers{d}] $\begin{pmatrix}
1 & 1 \\
0 & 1
\end{pmatrix}$
\task[\pers{e}] $\begin{pmatrix}
2 & 1 \\
1 & 2
\end{pmatrix}$
\task[\pers{f}] $\begin{pmatrix}
1 & 2 \\
2 & 1
\end{pmatrix}$
}
\end{exercicio}
\solucao{}
\begin{exercicio}
Decomponha num produto $T = SU$ de um operador auto-adjunto positivo definido $S$ e um operador ortogonal $U$ cada um dos operadores $T$ representados numa base ortonormal pelas seguintes matrizes:
\dividiritens{
\task[\pers{a}] $\begin{pmatrix}
2 & -1 \\
2 & 1
\end{pmatrix}$
\task[\pers{b}] $\begin{pmatrix}
1 & -4 \\
1 & 4
\end{pmatrix}$
\task[\pers{c}] $\begin{pmatrix}
4 & -2 &2 \\
4 & 4 & 1 \\
-2 & 4 & 2
\end{pmatrix}$
}
\end{exercicio}
\solucao{}

\begin{exercicio}
Determine a matriz de cada uma das formas bilineares abaixo, relativamente à base especificada:
\dividiritens{
\task[\pers{a}] $f \colon \mathbb{R}^4 \times \mathbb{R}^4 \to \mathbb{R}$ dada por $f(u,v) = \langle u,v \rangle,$ $\forall u,v \in \mathbb{R}^4,$ com relação à base
\[
\mathcal{B} = \{ (-2,0,3,1), (1,2,1,-1), (0,1,2,-1),(1,2,3,1) \}
\]
\task[\pers{b}] $f \colon \mathbb{C}^3 \times \mathbb{C}^3 \to \mathbb{C}$ dada por $f(u,v) = \langle u, a \rangle \langle b, v \rangle,$ com $a,b \in \mathbb{C}^3$ fixos, com relação à base canônica e $\mathbb{C}^3$ 
\task[\pers{c}] $f \colon \mathbb{R}^3 \times \mathbb{R}^3 \to \mathbb{C}$ dada por $f(u,v) = \langle T(u), v \rangle,$ onde $T \in \mathcal{L}(\mathbb{R}^3),$ com relação à base canônica de $\mathbb{R}^3.$  
}
\end{exercicio}
\solucao{
\dividiritens{
\task[\pers{a}] Sejam $u_1 = (-2,0,3,1), u_2 = (1,2,1,-1), u_3 = (0,1,2,-1)$ e $u_4 = (1,2,3,1).$ Tomando $A = [a_{ij}],$ com $a_{ij} = f(u_i, u_j),$ obtemos
\[
\begin{array}{cccc}
a_{11} = f(u_1,u_1) = 14 & a_{12} = f(u_1, u_2) = 0 & a_{13} = f(u_1, u_3) = 5 & a_{14} = f(u_1,u_4) = 8 \\
a_{21} = f(u_2,u_1) = 0 & a_{22} = f(u_2, u_2) = 7 & a_{23} = f(u_2, u_3) = 5 & a_{24} = f(u_2,u_4) = 7 \\
a_{31} = f(u_3,u_1) = 5 & a_{32} = f(u_3, u_2) = 5 & a_{33} = f(u_3, u_3) = 6 & a_{34} = f(u_3,u_4) = 7 \\
a_{41} = f(u_4,u_1) = 8 & a_{42} = f(u_4, u_2) = 7 & a_{43} = f(u_4, u_3) = 7 & a_{44} = f(u_4,u_4) = 15 \\
\end{array}
\]

Assim, 
\[
A =\begin{pmatrix}
14 &0 & 5 &  8 \\
0 &  7 &  5 &7 \\
5 &  5 & 6 & 7 \\
 8 &  7 &  7 &  15
\end{pmatrix}
\]
é a matriz de $f$ na base $\mathcal{B} = \{ u_1, u_2, u_3, u_4 \}.$
}
}


\end{document}