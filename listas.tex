\documentclass[11pt,a4paper]{article}
\usepackage{estilosexercicios}
\usepackage{hyperref}

%\usepackage[bottom=2cm,top=3cm,left=3cm,right=2cm]{geometry}
%\usepackage[utf8]{inputenc}
%Environments para esta lista
% ---------------------------------------------------
\definecolor{Floresta}{rgb}{0.13,0.54,0.13}
\newcommand{\exercicio}[1]{\section*{Exercício #1} \textcolor{blue}{\bf(#1)}}
\newcommand{\dividiritens}[1]{\begin{tasks}[counter-format={(tsk[a])},label-width=3.6ex, label-format = {\bfseries}, column-sep = {0pt}](1) #1 \end{tasks}}
\newcommand{\pers}[1]{\textcolor{Floresta}{$\negrito{(#1)} $}}

\newcommand{\solucao}[1]{
\textbf{\textcolor{red}{Solução:}} #1}
\newcommand{\figura}[1]{\input Arquivos_de_figs_Exercicios/#1} %Adicionar figuras do latex

% ---------------------------------------------------
\title{Álgebra Linear}
\author{MAT5730}
\date{2º semestre de 2019}

\begin{document}
\definecolor{Floresta}{rgb}{0.13,0.54,0.13}
\maketitle
\tableofcontents
\newpage
\begin{comment}

\begin{center}
\large\textbf{\textcolor{Floresta}{Lista 1}}\\
\end{center}

\end{comment}

\section{\textcolor{Floresta}{Lista 1}}


\exercicio{1}
Exemplo
\dividiritens{
\task[\pers{1}] Item 1
\task[\pers{2}] Item 2
\task[\pers{3}] Item 3
}

\solucao{
  \dividiritens{
\task[\pers{1}] Solução 1

\task[\pers{2}] Solução 2

\task[\pers{3}] Solução 3
}


}
DuglAS
\end{document} 