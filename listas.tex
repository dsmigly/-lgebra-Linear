\documentclass[11pt,a4paper]{article}
\usepackage{estilosexercicios}
\usepackage{hyperref}

%\usepackage[bottom=2cm,top=3cm,left=3cm,right=2cm]{geometry}
%\usepackage[utf8]{inputenc}
%Environments para esta lista
% ---------------------------------------------------
\definecolor{Floresta}{rgb}{0.13,0.54,0.13}
\newcommand{\exercicio}[1]{\subsection{Exercício #1} \textcolor{blue}{\bf(#1)}}
\newcommand{\dividiritens}[1]{\begin{tasks}[counter-format={(tsk[a])},label-width=3.6ex, label-format = {\bfseries}, column-sep = {0pt}](1) #1 \end{tasks}}
\newcommand{\pers}[1]{\textcolor{Floresta}{$\negrito{(#1)} $}}

\newcommand{\solucao}[1]{
\textbf{\\ \\ \textcolor{red}{Solução:}} #1}
\newcommand{\figura}[1]{\input Arquivos_de_figs_Exercicios/#1} %Adicionar figuras do latex

% ---------------------------------------------------
\title{Álgebra Linear}
\author{MAT5730}
\date{2º semestre de 2019}

\begin{document}
\definecolor{Floresta}{rgb}{0.13,0.54,0.13}
\maketitle
\tableofcontents
\newpage
\begin{comment}

\begin{center}
\large\textbf{\textcolor{Floresta}{Lista 1}}\\
\end{center}

\end{comment}

\section{\textcolor{Floresta}{Lista 1}}


\exercicio{1} Sejam $V$ um $K$-espaço vetorial e $W$ um subespaço de $V.$ Seja $S = \{v_i\}_{i\in I} \subset V$ tal que $S = \{v_i + W\}_{i\in I}$ é linearmente independente no espaço quociente $V/W.$ Mostre que se $A$ é um conjunto linearmente independente de $W$ então $S \cup A$ é um conjunto linearmente independente de $V.$
\solucao{}

\exercicio{2} Sejam $V$ um $K$-espaço vetorial e $W$ um subespaço de $V.$ Seja $S = \{v_i\}_{i \in I} \subset V$ tal que $S = \{v_i + W\}_{i \in I}$ gera o espaço quociente $V/W.$ Mostre que se $A$ é um conjunto gerador de
$W$ então $S \cup A$ é um conjunto gerador de $V$.
\solucao{}

\exercicio{3} Seja $V$ um $K$-espaço vetorial e sejam $U$ e $W$ subespaços de $V.$ Prove:
\dividiritens{
    \task[\pers{a}] O Segundo Teorema do Isomorfismo:
    \[
    \frac{U + W}{W} \cong \frac{U}{U \cap W}.
    \]
        \task[\pers{b}] O Terceiro Teorema do Isomorfismo: Se $U \subset W,$
        \[
        \frac{V}{W} \cong \frac{V/U}{W/U}
        \]
}

\solucao{}

\exercicio{4} Seja $V$ um $K$-espaço vetorial e sejam $U$ e $W$ subespaços de $V$ tais que $\dim (V/U) = m$ e $\dim (V/W) = n.$ Prove que $\dim (V/(U \cap W)) \le m + n.$

\solucao{}

\exercicio{5} Mostre que
\dividiritens{
    \task[\pers{a}] $W \oplus U= W^{\prime} \oplus U^{\prime} \ \mbox{e} \ W \cong W^{\prime} \nrightarrow U \cong U^{\prime}.$
   \task[\pers{b}] $V \cong V^{\prime}, V = W \oplus U \ \mbox{e} \ V^{\prime} = W \oplus U^{\prime} \nrightarrow U \cong U^{\prime}.$
}


\solucao{}

\exercicio{6} Seja $V$ um espaço vetorial e seja $W$ um subespaço de $V.$ Suponha que $V = V_1 \oplus \ldots \oplus V_n$ e $S = S_1 \oplus \ldots \oplus S_n,$ com $S_i \subseteq V_i$ subespaços de $V$ para todo $i = 1, \ldots, n.$ Mostre que
\[
\frac{V}{S} \cong \frac{V_1}{S_1} \oplus \ldots \oplus \frac{V_n}{S_n}.
\]
\solucao{}

\exercicio{7} Seja $V$ um $K$-espaço vetorial e seja $W$ um subespaço de $V.$ Seja $T \in \mathcal{L}(V)$ e defina $\overline{T} \colon V/W \to V/W$ por
\[
\overline{T}(v+ W) = T(v) + W, \mbox{ para todo } v + W \in V/W.
\]
\dividiritens{
    \task[\pers{a}]  Determine uma condição necessária e suficiente sobre $W$ para que $\overline{T}$ esteja bem definida.
   \task[\pers{b}]  Se $\overline{T}$ estiver bem definida, mostre que ela é linear e determine seu núcleo e sua imagem.
}
\solucao{}

\exercicio{8} Seja $T \in \mathcal{L}(\mathbb{R}^3)$ o operador linear definido por $T(x, y, z) = (x, x, x).$ Seja $T \colon \mathbb{R}^3/W \to \mathbb{R}^3/W$
tal que $\overline{T}((x, y, z) + W) = T(x, y, z) + W,$ em que $W = \mbox{ker } T.$ Descreva $\overline{T}.$

\solucao{}

\exercicio{9} Sejam $V$ e $U$ $K$-espaços vetoriais. Seja $W$ um subespaço de $V$ e $\pi \colon V \to V/W$ a projeção canônica. Mostre que a função $\mathcal{L}(V/W, U) \to \mathcal{L}(V, U),$ dada por $T \to T \circ \pi,$ é injetora.

\solucao{
%Temos a função\[\fullfunction{\varphi}{\mathcal{L}(V/W, U)}{ \mathcal{L}(V, U)}{T}{T \circ \pi}\]Para mostrar que $\varphi$ é injetora, precisamos verificar que, para $S, T \in \mathcal{L}(V/W, U),$ se $\varphi(S) = \varphi(T),$ então $S = T.$Verifiquemos primeiramente que $\varphi$ é uma transformação linear:\[(\varphi(S) - \varphi(T))(v) = \varphi(S - T)(v) = ((S - T) \circ \pi)(v) = (S-T)(\pi(v)) = S\]
}

\exercicio{10} Seja $V$ um $K$-espaço vetorial e seja $W$ um subespaço de $V.$ Mostre que $(V/W)^{*} \cong W^{0}$ e que $V^{*}/W^{0} \cong W^{*}.$

\solucao{Mostremos que $(V/W)^{*} \cong W^{0}.$ Para isso, a ideia será utilizar a aplicação canônica de $V$ em $V/W$ e sua transposta, e depois aplicar o Primeiro Teorema do Isomorfismo para obter o resultado desejado. Comecemos considerando a aplicação canônica
\[
\fullfunction{T}{V}{V/W}{v}{T(v) = v + W}
\]
Veja que $T$ é sobrejetora (isto é, $\mbox{Im } T = V/W$), e $\mbox{Ker } T = W.$ Consideremos a aplicação transposta
\[\fullfunction{T^t}{(V/W)^{*}}{V^{*}}{f}{T^t(f) = f \circ T}
\]
Das propriedades da transformação transposta, sabemos que
\[
\mbox{Ker } T^t = (\textcolor{Laranja}{\mbox{Im } T})^{0} = (\textcolor{Laranja}{V/W})^{0} = \{0\}
\]
\[
\mbox{Im } T^t = (\textcolor{Rosadif}{\mbox{Ker } T})^{0} = \textcolor{Rosadif}{W}^{0}
\]
Pelo Primeiro Teorema do Isomorfismo, temos que
\[
\frac{(V/W)^{*}}{\textcolor{Laranja}{\mbox{Ker } T^t}} \cong \textcolor{Rosadif}{\mbox{Im } T^t } \Rightarrow \frac{(V/W)^{*}}{\textcolor{Laranja}{\{ 0 \}}} \cong \textcolor{Rosadif}{W^{0}} \Rightarrow \boxed{ (V/W)^{*} \cong W^0}
\]

Mostremos agora que $V^{*}/W^{0} \cong W^{*}.$ Utilizaremos a mesma estratégia, mas considerando agora a inclusão. Tome a inclusão de $W$ em $V,$ isto é:
\[
\fullfunction{\iota}{W}{V}{w}{\iota(w) = w}
\]
Note que $\mbox{Ker }\iota = \{ 0 \}$ e $\mbox{Im } \iota = W.$
Seja
\[
\fullfunction{\iota^t}{V^{*}}{W^{*}}{f}{\iota(f) = f \circ \iota}
\]
a transposta de $\iota.$ Observe que
\[
\mbox{Ker } \iota^t = (\textcolor{Laranja}{\mbox{Im } \iota})^{0} = \textcolor{Laranja}{W}^{0} 
\]
\[
\mbox{Im } \iota^t = (\textcolor{Rosadif}{\mbox{Ker } \iota})^{0} = \textcolor{Rosadif}{\{0\}}^{0} = W^{*}\]
Pelo Primeiro Teorema do Isomorfismo,
\[\frac{V^{*}}{\textcolor{Laranja}{\mbox{Ker } \iota^t}} \cong \textcolor{Rosadif}{\mbox{Im }\iota^t }\Rightarrow \frac{V^{*}}{\textcolor{Laranja}{W^{0}}} \cong \textcolor{Rosadif}{W^{*}} \Rightarrow \boxed{ V^{*}/W^0 \cong W^{*}}\]
}

\exercicio{11} Sejam $A,B,C \in \mathcal{M}_n(K).$ Prove que
\[
\det \left[ \begin{array}{cc} 0 & C \\ A & B \end{array} \right] = (-1)^n \det(A) \det(C).
\]
\solucao{}

\exercicio{12} Calcule o determinante da matriz de Vandermonde, isto é, prove que
\[
\det \left[ \begin{array}{cccc} 1 & 1 & \ldots & 1 \\ c_1 & c_2 & \ldots & c_n \\ \vdots & \vdots & \ddots & \vdots \\ c_1^{n-1} & c_2^{n-1} & \ldots & c_n^{n-1} \end{array} \right] = \prod\limits_{1 \le i < j \le n} (c_j - c_i)
\]
\solucao{
Vamos provar o resultado por indução sobre $n \ge 2.$ Para $n = 2,$ é fácil ver que
\[
\det \left[ \begin{array}{cc} 1 & 1 \\ c_1 & c_2 \end{array} \right] = c_2 - c_1 = \prod\limits_{1 \le i < j \le 2} (c_j - c_i)
\]
Assuma o resultado válido para $n - 1,$ ou seja, 
\[
\det \left[ \begin{array}{cccc} 1 & 1 & \ldots & 1 \\ c_1 & c_2 & \ldots & c_n \\ \vdots & \vdots & \ddots & \vdots \\ c_1^{n-2} & c_2^{n-2} & \ldots & c_n^{n-2} \end{array} \right] = \prod\limits_{1 \le i < j \le n-1} (c_j - c_i) = \prod\limits_{1 \le i < j \le n-1} (c_j - c_i)
\]
Provemos para a matriz $n \times n.$ Utilizando a matriz transposta, vamos aplicar operações nas colunas da matriz de modo a obter zeros na primeira linha. Para isso, vamos multiplicar cada coluna $C_i$ por $-c_1$ e somaremos com a coluna $C_{i+1},$ obtendo
\[\begin{bmatrix} 
\textcolor{Laranja}{1} & \textcolor{Verde}{c_1} & \textcolor{Blue}{c_1^2} & \textcolor{RawSienna}{\dots} & \textcolor{Purple}{c_1^{n-1}}\\
\textcolor{Laranja}{1} & \textcolor{Verde}{c_2} & \textcolor{Blue}{c_2^2} & \textcolor{RawSienna}{\dots}  & \textcolor{Purple}{c_2^{n-1}}\\ 
\textcolor{Laranja}{1} & \textcolor{Verde}{c_3} & \textcolor{Blue}{c_3^2} & \textcolor{RawSienna}{\dots}  &  \textcolor{Purple}{c_3^{n-1}}\\ 
\textcolor{Laranja}{\vdots} & \textcolor{Verde}{\vdots} & \vdots &\textcolor{RawSienna}{\ddots}  &\vdots \\ 
\textcolor{Laranja}{1} & \textcolor{Verde}{c_n} & \textcolor{Blue}{c_n^2} & \textcolor{RawSienna}{\dots}  &  \textcolor{Purple}{c_n^{n-1}}\\ \end{bmatrix} \xrightarrow{C_{i+1} = C_{i+1} - \textcolor{red}{c_1}C_i} \begin{bmatrix} \textcolor{Laranja}{1} & \textcolor{Verde}{c_1} - \textcolor{red}{c_1}\textcolor{Laranja}{1}  & \textcolor{Blue}{c_1^2} - \textcolor{red}{c_1}\textcolor{Verde}{c_1} & \textcolor{RawSienna}{\dots}  & \textcolor{Purple}{c_1^{n-1}} - \textcolor{red}{c_1}\textcolor{RawSienna}{c_1^{n-2}}\\ 
\textcolor{Laranja}{1} & \textcolor{Verde}{c_2}  - \textcolor{red}{c_1}\textcolor{Laranja}{1}  &  \textcolor{Blue}{c_2^2} - \textcolor{red}{c_1}\textcolor{Verde}{c_2}  & \textcolor{RawSienna}{\dots}  & \textcolor{Purple}{c_2^{n-1}} - \textcolor{red}{c_1}\textcolor{RawSienna}{c_2^{n-2}}\\ 
\textcolor{Laranja}{1} &  \textcolor{Verde}{c_3}  - \textcolor{red}{c_1}\textcolor{Laranja}{1} & \textcolor{Blue}{c_3^2} - \textcolor{red}{c_1}\textcolor{Verde}{c_3}  & \textcolor{RawSienna}{\dots}  & \textcolor{Purple}{c_3^{n-1}} - \textcolor{red}{c_1}\textcolor{RawSienna}{c_3^{n-2}}\\
\vdots & \vdots & \vdots & \ddots  &\vdots \\ 
\textcolor{Laranja}{1} &  \textcolor{Verde}{c_n}  - \textcolor{red}{c_1}\textcolor{Laranja}{1}  & \textcolor{Blue}{c_n^2} - \textcolor{red}{c_1}\textcolor{Verde}{c_n} & \textcolor{RawSienna}{\dots}  & \textcolor{Purple}{c_n^{n-1}} - \textcolor{red}{c_1}\textcolor{RawSienna}{c_n^{n-2}}\\ \end{bmatrix}=\]
\[\begin{bmatrix} 1 & 0 & 0 & \dots & 0\\ 1 & c_2-c_1 & c_2(c_2-c_1) & \dots & c_2^{n-2}(c_2-c_1)\\ 1 & c_3-c_1 & c_3(c_3-c_1) & \dots & c_3^{n-2}(c_3-c_1)\\ \vdots & \vdots & \vdots & \ddots &\vdots \\ 1 & c_n-c_1 & c_n(c_n-c_1) & \dots & c_n^{n-2}(c_n-c_1)\\ \end{bmatrix}\]

Utilizando o Teorema de Laplace, temos que
\[
\det \left[\begin{array}{c|cccc} 1 & 0 & 0 & \dots & 0\\ \hline 1 & c_2-c_1 & c_2(c_2-c_1) & \dots & c_2^{n-2}(c_2-c_1)\\ 1 & c_3-c_1 & c_3(c_3-c_1) & \dots & c_3^{n-2}(c_3-c_1)\\ \vdots & \vdots & \vdots & \ddots &\vdots \\ 1 & c_n-c_1 & c_n(c_n-c_1) & \dots & c_n^{n-2}(c_n-c_1)\\ \end{array}\right] =\]\[ \det \left[\begin{array}{cccc}   c_2-c_1 & c_2(c_2-c_1) & \dots & c_2^{n-2}(c_2-c_1)\\  c_3-c_1 & c_3(c_3-c_1) & \dots & c_3^{n-2}(c_3-c_1)\\  \vdots & \vdots & \ddots &\vdots \\ c_n-c_1 & c_n(c_n-c_1) & \dots & c_n^{n-2}(c_n-c_1)\\ \end{array}\right]
\]
Como cada linha está multiplicada por $c_i - c_1,$ por propriedades do determinante, temos que
\[
\det \left[\begin{array}{cccc}   \textcolor{Verde}{c_2-c_1} & c_2\textcolor{Verde}{(c_2-c_1)} & \dots & c_2^{n-2}\textcolor{Verde}{(c_2-c_1)}\\  \textcolor{Blue}{c_3-c_1} & c_3\textcolor{Blue}{(c_3-c_1)} & \dots & c_3^{n-2}\textcolor{Blue}{(c_3-c_1)}\\  \vdots & \vdots & \ddots &\vdots \\ \textcolor{Red}{c_n-c_1} & c_n\textcolor{Red}{(c_n-c_1)} & \dots & c_n^{n-2}\textcolor{Red}{(c_n-c_1)}\\ \end{array}\right] =\]\[
\textcolor{Verde}{(c_2-c_1)}\textcolor{Blue}{(c_3-c_1)} \cdot \ldots \cdot  \textcolor{Red}{(c_n-c_1)}  \det \left[\begin{array}{ccccc}   1& c_2 & c_2^2 & \dots & c_2^{n-2}\\  1 & c_3 & c_3^2 & \dots & c_3^{n-2} \\  1 & c_4 & c_4^2 & \dots & c_4^{n-2}\\  \vdots & \vdots & \ddots &\vdots \\ 1 & c_n & c_n^2 & \dots & c_n^{n-2}\\ \end{array}\right] = \]\[
\prod\limits_{j = 2}^n (c_j - c_1)  \det \left[\begin{array}{ccccc}   1& c_2 & c_2^2 & \dots & c_2^{n-2}\\  1 & c_3 & c_3^2 & \dots & c_3^{n-2} \\  1 & c_4 & c_4^2 & \dots & c_4^{n-2}\\  \vdots & \vdots & \ddots &\vdots \\ 1 & c_n & c_n^2 & \dots & c_n^{n-2}\\ \end{array}\right] 
\]
Como a matriz resultante tem tamanho $n-1 \times n-1,$ da hipótese de indução, vem
\[
 \det \left[\begin{array}{ccccc}   1& c_2 & c_2^2 & \dots & c_2^{n-2}\\  1 & c_3 & c_3^2 & \dots & c_3^{n-2} \\  1 & c_4 & c_4^2 & \dots & c_4^{n-2}\\  \vdots & \vdots & \ddots &\vdots \\ 1 & c_n & c_n^2 & \dots & c_n^{n-2}\\ \end{array}\right] =  \prod\limits_{2 \le i < j \le n} (c_j - c_i).
\]
Daí,
\[
\left(\prod\limits_{j = 2}^n (c_j - c_1) \right) \textcolor{red}{\det \left[\begin{array}{ccccc}   1& c_2 & c_2^2 & \dots & c_2^{n-2}\\  1 & c_3 & c_3^2 & \dots & c_3^{n-2} \\  1 & c_4 & c_4^2 & \dots & c_4^{n-2}\\  \vdots & \vdots & \ddots &\vdots \\ 1 & c_n & c_n^2 & \dots & c_n^{n-2}\\ \end{array}\right]} = \]\[\left(\prod\limits_{j = 2}^n (c_j - c_1) \right) \textcolor{red}{\left(\prod\limits_{2 \le i < j \le n} (c_j - c_i) \right)} = \prod\limits_{1 \le i < j \le n} (c_j - c_i) 
\]
Assim, segue o resultado.
}

\exercicio{13} Mostre que
\[
\det \left[ \begin{array}{cccc} a & -b & -c & -d \\ b & a & -d & c \\ c & d & a & -b \\ d & -c & b & a \end{array} \right] = (a^2 + b^2 + c^2 + d^2)^2
\]
\solucao{
Primeiramente, vamos mostrar que, para $A, B \in \mathcal{M}_n(\mathbb{C}),$ temos que
\[
\det \begin{bmatrix} A & -B \\ B & A \end{bmatrix} = \abs{\det(A + Bi)}^2
\]
De fato:
\[\det \left(\begin{array}{cc} A & -B\\B & A\end{array}\right)=\det\left(\begin{array}{cc} A-iB & -B\\B+iA & A\end{array}\right)= \det\left(\begin{array}{cc} A - iB & -B\\i(A - iB) & A\end{array}\right)= \]\[\det\left(\begin{array}{cc} A - iB & -B\\i  (A - iB) -i (A - iB) & A +i B\end{array}\right)= \det \left(\begin{array}{cc}  A - iB & -B\\0 &  A + iB\end{array}\right)=\abs{\det(A + Bi)}^2\]
Portanto, escrevendo
\[
A = \begin{bmatrix} a & -b \\ b & a \end{bmatrix} \quad \mbox{ e } \quad B = \begin{bmatrix} c & d \\ d & -c \end{bmatrix},
\]
segue que 
\[
\det \left[ \begin{array}{cccc} a & -b & -c & -d \\ b & a & -d & c \\ c & d & a & -b \\ d & -c & b & a \end{array} \right] = \det \begin{bmatrix} A & -B \\ B & A \end{bmatrix} = \abs{\det(A + Bi)}^2.\]
Como 
\[
A + Bi =  \begin{bmatrix} a & -b \\ b & a \end{bmatrix} + \begin{bmatrix} c & d \\ d & -c \end{bmatrix}i =  \begin{bmatrix} a + ci & -b + di \\ b+di & a - ci \end{bmatrix},
\]
temos que
\[
\abs{\det(A + Bi)}^2 = \abs{\det \begin{bmatrix} a + ci & -b + di \\ b+di & a - ci \end{bmatrix}}^2 = \abs{(a+ci)(a-ci) - (di - b)(di+b)}^2  = \]\[\abs{a^2 + c^2 - (- b^2 - d^2)}^2 = \abs{a^2 + c^2 + b^2 + d^2}^2 = (a^2 + b^2 + c^2 + d^2)^2
\]
}


\exercicio{14} Sejam $A, B \in \mathcal{M}_n(K).$ Mostre que se $A$ é inversível então existem no máximo $n$ escalares $c$
tais que $cA + B$ não é inversível. 

\solucao{}

\exercicio{15} Sejam $A, B, C, D\in \mathcal{M}_n(K)$ com $D$ inversível.

\dividiritens{
    \task[\pers{a}] Mostre que
\[\det \left[ \begin{array}{cc} A & B \\ C & D \end{array} \right] = \det(AD - BD^{-1}CD)\]
   \task[\pers{b}] Se $CD = DC,$ mostre que
\[\det \left[ \begin{array}{cc} A & B \\ C & D \end{array} \right] = \det(AD - BC).\] O que acontece quando $D$ não é inversível?
\task[\pers{c}] Se $DB = BD,$ calcule $\det \left[ \begin{array}{cc} A & B \\ C & D \end{array} \right].$
}

\solucao{Pelo Teorema de Binet, sabemos que o determinante de um produto de duas matrizes quadradas é o produto de seus determinantes, ou seja, se $X, Y \in \mathcal{M}_n(K),$ então
    \[
    \det (X) \det(Y) = \det(XY)
    \]
    Além disso, lembramos que, para $U, V, X, Y$ $n \times n,$ temos
    \[
    \det \begin{bmatrix} U & 0 \\ X & Y \end{bmatrix} = \det U \det Y
    \]
    e
     \[
    \det \begin{bmatrix} U & V \\ 0 & Y \end{bmatrix} = \det U \det Y
    \]
    Feitas essas observações, estamos aptos a resolver a questão.
\dividiritens{
    \task[\pers{a}] Para obter o resultado desejado, a ideia será multiplicar a matriz em questão por uma matriz conveniente cujo determinante é $1.$ Dessa forma, utilizando as observações acima, sendo $I_n$ a notação para a matriz identidade $n \times n,$ e lembrando que $D$ é invertível, temos que
\[ \left( \begin{array}{cc} A & B \\ C & D \end{array} \right)  \left( \begin{array}{cc} I_n & 0 \\ -D^{-1}C & I_n \end{array} \right) = \left( \begin{array}{cc} A - BD^{-1}C & B \\ 0 & D \end{array} \right)
    \]
Calculando os determinantes, vem
\[
\det \left( \left[ \begin{array}{cc} A & B \\ C & D \end{array} \right] \left[ \begin{array}{cc} I_n & 0 \\ -D^{-1}C & I_n \end{array} \right] \right) = \det \left( \begin{array}{cc} A - BD^{-1}C & B \\ 0 & D \end{array} \right) \Rightarrow
\]
\[
\det \left( \left[ \begin{array}{cc} A & B \\ C & D \end{array} \right] \right) \cdot \textcolor{Blue}{ \det \left(\left[ \begin{array}{cc} I_n & 0 \\ -D^{-1}C & I_n \end{array} \right] \right) } = \textcolor{Verde}{ \det \left( \begin{array}{cc} A - BD^{-1}C & B \\ 0 & D \end{array} \right)} \Rightarrow
\]
\[
\det \left( \left[ \begin{array}{cc} A & B \\ C & D \end{array} \right] \right) \cdot \textcolor{Blue}{ \det I_n  \cdot \det I_n} = \textcolor{Verde}{ \det \left(A - BD^{-1}C \right) \det (D)} \Rightarrow
\]
\[
\det \left( \left[ \begin{array}{cc} A & B \\ C & D \end{array} \right] \right) \cdot \det (I_n I_n) =  \det \left((A - BD^{-1}C)D \right) \Rightarrow \]\[\det \left( \left[ \begin{array}{cc} A & B \\ C & D \end{array} \right] \right) \cdot \det (I_n) =  \det (AD - BD^{-1}CD) \Rightarrow 
\]
\[
\boxed{\det \left[ \begin{array}{cc} A & B \\ C & D \end{array} \right] =  \det \left(AD - BD^{-1}CD \right)}
\]
    \task[\pers{b}] Utilizando as observações acima, sendo $I_n$ a notação para a matriz identidade $n \times n,$ e usando o fato de que $CD = DC,$ temos que
    \[
   \left( \begin{array}{cc} A & B \\ C & D \end{array} \right)  \left( \begin{array}{cc} D & 0 \\ -C & I_n \end{array} \right) = \left( \begin{array}{cc} AD - BC & B \\ \textcolor{red}{CD - DC} & D \end{array} \right) = \left( \begin{array}{cc} AD - BC & B \\ \textcolor{red}{0} & D \end{array} \right) 
    \]
    Como $D$ é invertível, temos $\det D \neq 0.$ Portanto, segue que 
    \[
       \det \left(\left[ \begin{array}{cc} A & B \\ C & D \end{array} \right]  \left[ \begin{array}{cc} D & 0 \\ -C & I_n \end{array} \right] \right) = \det \left( \begin{array}{cc} AD - BC & B \\ 0 & D \end{array} \right) \Rightarrow \]
       \[\det \left(\left[ \begin{array}{cc} A & B \\ C & D \end{array} \right]\right)  \cdot \textcolor{Blue}{\det \left( \left[ \begin{array}{cc} D & 0 \\ -C & I_n \end{array} \right] \right)} = \textcolor{Verde}{ \det \left( \begin{array}{cc} AD - BC & B \\ 0 & D \end{array} \right)} \Rightarrow \]
       \[\det \left(\left[ \begin{array}{cc} A & B \\ C & D \end{array} \right]\right)  \cdot \textcolor{Blue}{\det(D) \det(I_n)} = \textcolor{Verde}{ \det (AD - BC) \det(D)}\Rightarrow \]
       \[   \det \left[ \begin{array}{cc} A & B \\ C & D \end{array} \right]= \det (AD - BC) \det(D)\cdot \frac{1}{\det(D)} \Rightarrow  \]
       \[  \boxed{\det \left[ \begin{array}{cc} A & B \\ C & D \end{array} \right] = \det (AD - BC)  } \]
           \task[\pers{c}] Para resolver este item, vamos utilizar as propriedades das matrizes transpostas. Lembrando que, se $X, Y \in \mathcal{M}_n(K),$ então
           \begin{itemize}
           \item $(X^t)^t = X;$
               \item $(X + Y)^t = X^t + Y^t;$
               \item $(XY)^t = Y^tX^t;$
               \item $\det(X^t) = \det(X).$
           \end{itemize}
           de posse dessas propriedades, observe que
           \[
            \left[ \begin{array}{cc} A & B \\ C & D \end{array} \right]^t =   \left[ \begin{array}{cc} A^t & C^t \\ B^t & D^t \end{array} \right]
           \]
           Daí, utilizando a notação $I_n$ para a matriz identidade $n \times n,$ e usando o fato de que $DB = BD,$
               \[
   \left( \begin{array}{cc} A^t & C^t \\ B^t & D^t \end{array} \right)  \left( \begin{array}{cc} D^t & 0 \\ -B^t & I_n \end{array} \right) = \left( \begin{array}{cc} A^tD^t - B^tC^t & C^t \\ B^tD^t - D^tB^t & D^t \end{array} \right) = \left( \begin{array}{cc} (DA)^t - (CB)^t & C^t \\ (DB)^t - (BD)^t & D^t \end{array} \right) = \]\[\left( \begin{array}{cc} (DA - CB)^t & C^t \\ \textcolor{red}{(DB - BD)^t} & D^t \end{array} \right) = \left( \begin{array}{cc} (DA - CB)^t & C^t \\ \textcolor{red}{0} & D^t \end{array} \right) 
    \]
    Novamente, sendo $D$ invertível, então $D^t$ também é invertível. Logo, temos
                   \[
   \det\left( \left[\begin{array}{cc} A^t & C^t \\ B^t & D^t \end{array} \right]  \left[ \begin{array}{cc} D^t & 0 \\ -B^t & I_n \end{array} \right] \right) = \det \left( \begin{array}{cc} (DA - CB)^t & C^t \\ 0 & D^t \end{array} \right) \Rightarrow
    \]
    \[
       \det\left( \left[\begin{array}{cc} A^t & C^t \\ B^t & D^t \end{array} \right] \right) \textcolor{Blue}{\det \left( \left[ \begin{array}{cc} D^t & 0 \\ -B^t & I_n \end{array} \right] \right)} = \textcolor{Verde}{\det \left( \begin{array}{cc} (DA - CB)^t & C^t \\ 0 & D^t \end{array} \right)} \Rightarrow
    \]
        \[
       \det\left( \left[\begin{array}{cc} A^t & C^t \\ B^t & D^t \end{array} \right] \right) \textcolor{Blue}{\det (D^t) \det(I_n)} = \textcolor{Verde}{\det \left((DA - CB)^t \right) \det\left(D^t \right)} \Rightarrow
    \]
            \[
       \det \left[\begin{array}{cc} A^t & C^t \\ B^t & D^t \end{array} \right] = \det \left((DA - CB)^t \right) \det\left(D^t \right)\cdot \frac{1}{\det\left(D^t \right)} \Rightarrow
    \]
                \[
       \det \left[\begin{array}{cc} A^t & C^t \\ B^t & D^t \end{array} \right] = \det \left((DA - CB)^t \right) \Rightarrow \boxed{  \det \left[\begin{array}{cc} A^t & C^t \\ B^t & D^t \end{array} \right] = \det \left(DA - CB \right)}
    \]
    }
    }

\exercicio{16} Seja $A \in \mathcal{M}_{m \times n}(K).$ Prove que
\[
\det(I_m + AA^t) = \det(I_n + A^tA)
\]
\solucao{}

\exercicio{17} Seja $\sigma \in S_n$ e defina 
\[
\fullfunction{T_\sigma}{K^n}{K^n}{e_i}{T_\sigma(e_i) = e_{\sigma(i)}},
\]
para $i = \{ 1, 2, \ldots, n \}$ e $\{e_1, e_2, \ldots, e_n\}$ é a base canônica de $K^n.$ Calcule $\det(T_\sigma).$
\solucao{}

\exercicio{18} Seja $C \in \mathcal{M}_n(K)$ a matriz
\[
\left[ \begin{array}{cccccc} x & 0 & 0 & \ldots & 0 & c_0 \\ -1 & x & 0 & \ldots & 0 & c_1 \\0 & -1 & x & \ldots & 0 & c_2 \\ \vdots & \vdots & \vdots & \ddots & \vdots & \vdots \\  0 & 0 & 0 & \ldots & x & c_{n-2}  \\  0 & 0 & 0 & \ldots & -1 & x + c_{n-1}\end{array} \right]
\]

Prove que $\det C = x^n + c_{n-1}x^{n-1} + \ldots + c_1x + c_0.$
\solucao{ Vamos provar o resultado por indução sobre $n \ge 2.$

Para $n = 2,$ temos que\[ C = \left[ \begin{array}{cc} x & c_0 \\ -1 & x+c_1 \end{array} \right].\] Portanto,
\[
\det C = x(x+c_1) + c_0 = x^2 + c_1x + c_0.
\]
Seja agora $n > 2$ e admita que o resultado é verdadeiro para matrizes $n - 1 \times n-1$ desse tipo.

Usando o desenvolvimento de $\det C$ por Laplace, pela primeira linha, temos que
\[
\det \left[ \begin{array}{c|cccc|c} \textcolor{red}{x} & 0 & 0 & \ldots & 0 & \textcolor{Verde}{c_0} \\ \hline \textcolor{red}{-1} & \textcolor{blue}{x} & \textcolor{blue}{0} & \textcolor{blue}{\ldots} & \textcolor{blue}{0} & \textcolor{Verde}{c_1} \\ \textcolor{red}{0} & \textcolor{blue}{-1} & \textcolor{blue}{x} & \textcolor{blue}{\ldots} & \textcolor{blue}{0} & \textcolor{Verde}{c_2} \\ \vdots & \textcolor{blue}{\vdots} & \textcolor{blue}{\vdots} & \textcolor{blue}{\ddots} & \textcolor{blue}{\vdots} & \textcolor{Verde}{\vdots} \\  \textcolor{red}{0} & \textcolor{blue}{0} & \textcolor{blue}{0} & \textcolor{blue}{\ldots} & \textcolor{blue}{x} & \textcolor{Verde}{c_{n-2}}  \\  \textcolor{red}{0} & \textcolor{blue}{0} & \textcolor{blue}{0} & \textcolor{blue}{\ldots} & \textcolor{blue}{-1} & \textcolor{Verde}{x + c_{n-1}}\end{array} \right] = \]\[
\textcolor{red}{x} \cdot \det
\left[ \begin{array}{ccccc} \textcolor{blue}{x} & \textcolor{blue}{0} & \textcolor{blue}{\ldots} & \textcolor{blue}{0} & \textcolor{Verde}{c_1} \\ \textcolor{blue}{-1} & \textcolor{blue}{x} & \textcolor{blue}{\ldots} & \textcolor{blue}{0} & \textcolor{Verde}{c_2} \\ \textcolor{blue}{0} & \textcolor{blue}{-1} & \textcolor{blue}{\ldots} & \textcolor{blue}{0} & \textcolor{Verde}{c_3} \\ \textcolor{blue}{\vdots} & \textcolor{blue}{\vdots} & \textcolor{blue}{\ddots} & \textcolor{blue}{\vdots} & \textcolor{Verde}{\vdots} \\  \textcolor{blue}{0} & \textcolor{blue}{0} & \textcolor{blue}{\ldots} & \textcolor{blue}{x} & \textcolor{Verde}{c_{n-2}}  \\  \textcolor{blue}{0} & \textcolor{blue}{0} & \ldots & \textcolor{blue}{-1} & \textcolor{Verde}{x + c_{n-1}}\end{array} \right] + (-1)^{n+1} \textcolor{Verde}{c_0} \det
\left[ \begin{array}{cccccc} \textcolor{red}{-1} & \textcolor{blue}{x} &\textcolor{blue}{0} & \textcolor{blue}{\ldots} & \textcolor{blue}{0} & \textcolor{blue}{0}  \\ \textcolor{red}{0} & \textcolor{blue}{-1} & \textcolor{blue}{x} & \textcolor{blue}{\ldots} & \textcolor{blue}{0} & \textcolor{blue}{0} \\ \textcolor{red}{0} & \textcolor{blue}{0} & \textcolor{blue}{-1} & \textcolor{blue}{\ldots} & \textcolor{blue}{0} & \textcolor{blue}{0} \\ \textcolor{red}{\vdots} & \textcolor{blue}{\vdots} & \textcolor{blue}{\vdots} & \textcolor{blue}{\ddots} & \textcolor{blue}{\vdots} & \textcolor{blue}{\vdots} \\  \textcolor{red}{0} & \textcolor{blue}{0} & \textcolor{blue}{0} & \textcolor{blue}{\ldots} & \textcolor{blue}{-1} & \textcolor{blue}{x}  \\  \textcolor{red}{0} & \textcolor{blue}{0} & \textcolor{blue}{0} & \textcolor{blue}{\ldots} & \textcolor{blue}{0} & \textcolor{blue}{-1} \end{array} \right]
\]
Pela hipótese de indução, segue que\[ \det C = x(x^{n-1} + c_{n-1}x^{n-2} + \ldots + c_2x + c_1) + (-1)^{n+1} c_0 (-1)^{n-1} = x^n + c_{n-1}x^{n-1} + \ldots + c_1x + c_0,\]
como queríamos.



}

\exercicio{19} Seja $K$ um corpo e $A_1, \ldots, A_n$ matrizes quadradas sobre $K$. Seja $B$ a matriz triangular por blocos 
\[
 \left[ \begin{array}{cccc} A_1 & * & \ldots & * \\ 0 & A_2 & \ddots & \vdots \\ \vdots & \ddots & \ddots & * \\ 0 & \ldots & 0 & A_n \end{array} \right]
\]

Mostre que $\det B = \det(A_1)\det(A_2)\ldots \det(A_n).$
\solucao{}

\exercicio{20} Seja $K$ um corpo e $a,b,c,d,e,f,g \in K.$ Mostre que
\[\det \left[ \begin{array}{ccc} a & b & b \\ c & d & e \\ f & g & g \end{array} \right] + \det \left[ \begin{array}{ccc} a & b & b \\ e & c & d \\ f & g & g \end{array} \right] + \det \left[ \begin{array}{ccc} a & b & b \\ d & e & c \\  f & g & g \end{array} \right] = 0\]
\solucao{
Temos que o determinante é uma forma $3$-linear das linhas da matriz, então:
\[
\det \left[ \begin{array}{ccc} a & b & b \\ c & d & e \\ f & g & g \end{array} \right] + \det \left[ \begin{array}{ccc} a & b & b \\ e & c & d \\ f & g & g \end{array} \right] + \det \left[ \begin{array}{ccc} a & b & b \\ d & e & c \\  f & g & g \end{array} \right]  = \det \left[ \begin{array}{ccc} a & b & b \\ c+d+e & d+c+e & e+d+c \\  f & g & g \end{array} \right]
\]
Note que a segunda e a terceira coluna são iguais. Como o determinante é $3$-linear e alternado nas colunas da matriz, segue que
\[
\det \left[ \begin{array}{ccc} a & b & b \\ c+d+e & d+c+e & e+d+c \\  f & g & g \end{array} \right] = 0.
\]
}

\exercicio{21} Sabendo que os números inteiros $23028, 31882, 86469, 6327$ e $61902$ são todos múltiplos de $19,$ mostre que o número inteiro \[\det  \left[ \begin{array}{ccccc} 2 & 3 &0  & 2 & 8\\ 3 & 1 & 8 & 8 & 2 \\ 8 & 6 & 4 & 6 & 9 \\ 0 & 6 & 3 & 2 & 7 \\ 6 & 1 & 9 & 0 & 2 \end{array} \right]\] é múltiplo de 19.

\solucao{}


\exercicio{22} Seja $K$ corpo e $a,b,c \in K.$ Usando a matriz $ \left[ \begin{array}{ccc} b & c & 0\\ a & 0 & c \\ 0 & a & b \end{array} \right],$ calcule 
\[\det \left[ \begin{array}{ccc} b^2 + c^2 & ab & ac\\ab & a^2 + c^2 & bc \\ ac & bc & a^2 + b^2 \end{array} \right]\]
\solucao{}


\exercicio{23} Seja $K$ um corpo e $n$ um inteiro positivo. Dadas matrizes $A, B \in \mathcal{M}_n(K)$ mostre que
\[
\det \left[ \begin{array}{cc} A & B \\B & A \end{array} \right] = \det(A+B) \det(A-B)
\]

\solucao{}

\exercicio{24} Seja $K$ um corpo e $V$ um espaço vetorial de dimensão finita $n.$ Sejam $B = (e_1,\ldots ,e_n)$ e $C = (d_1, \ldots , d_n)$ duas bases de $V.$ Sejam $\varphi$ a única forma $n$-linear tal que $\varphi(e_1, \ldots ,e_n) = 1$ e $\psi$ a única forma $n$-linear tal que $\psi(d_1, \ldots, d_n) = 1.$ Qual o valor de $\psi(e_1, \ldots ,e_n)$ e de $\varphi(d_1, \ldots, d_n)?$ Use isso para dar uma relação entre $\psi$ e $\varphi.$
\solucao{}

\exercicio{25} Seja $K$ um corpo, $n$ um inteiro positivo e $K_n[t]$ o conjunto de polinômios de grau menor ou igual que $n$ com coeficientes em $K.$ Sejam $t_1,  \ldots, t_{n+1} \in K$ dois a dois distintos. Considere para $i = 1, \ldots, n+1$ as funções de avaliação
\[\fullfunction{\tau_i}{K_n[t]}{K}{p(t)}{\tau_i(p(t)) = p(t_i)}\]

\dividiritens{
    \task[\pers{a}] Mostre que $\mathcal{B} = \{ \tau_1, \ldots, \tau_{n+1}\}$ é base de $K_n[t]^{*}.$ (Sugestão: use o exercício 12.)
   \task[\pers{b}] Mostre que os \emph{polinômios de Lagrange}
\[L_i(t) = \prod\limits_{j \neq i} \frac{t-t_j}{t_i - t_j}, i = 1, \ldots, n+1,\] formam uma base dual de $\mathcal{B}.$
\task[\pers{c}] Mostre que para quaisquer $a_1,\ldots , a_{n+1} \in K$ existe um único polinômio $p(t)$ de grau menor o igual que $n$ tal que $p(t_i) = a_i,$ para $i=1, \ldots, n+1.$ (O resultado do item (c) é a conhecida \emph{Fórmula de Interpolação de Lagrange})
}
\solucao{
\dividiritens{
    \task[\pers{a}] Como $K_n[t]$ é um $K$-espaço vetorial de dimensão finita, temos que $\dim K_n[t]^{*} = \dim K_n[t] = n+ 1.$ Logo, para provar que $\mathcal{B}$ é base, basta mostrar que $\mathcal{B}$ é LI. 
    
    Sejam $\alpha_1, \ldots, \alpha_{n+1} \in K$ tais que
    \[
    \sum\limits_{i = 1}^{n+1} \alpha_i \tau_i = \alpha_1 \tau_1 + \ldots + \alpha_{n+1} \tau_{n+1} = 0
    \]
    Vamos mostrar que $\alpha_i = 0 \ \forall i \in \{1, \ldots, n+1 \}.$ Avaliemos $ \sum\limits_{i = 1}^{n+1} \alpha_i \tau_i $ em $1,t, \ldots, t^n:$
    \[   \left\{ \begin{array}{l}      \sum\limits_{i = 1}^{n+1} \alpha_i \textcolor{red}{\tau_i(1)} = \alpha_1 \textcolor{red}{\tau_1(1)} + \ldots + \alpha_{n+1} \textcolor{red}{\tau_{n+1}(1)} = 0 \\
            \sum\limits_{i = 1}^{n+1} \alpha_i \textcolor{Green}{\tau_i(t)} = \alpha_1 \textcolor{Green}{\tau_1(t)} + \ldots + \alpha_{n+1} \textcolor{Green}{\tau_{n+1}(t)} = 0 \\
            \vdots \\  \sum\limits_{i = 1}^{n+1} \alpha_i \textcolor{Blue}{\tau_i(t^n)} = \alpha_1 \textcolor{Blue}{\tau_1(t^n)} + \ldots + \alpha_{n+1} \textcolor{Blue}{\tau_{n+1}(t^n)} = 0 \\
    \end{array} \right. \Rightarrow    \left\{ \begin{array}{l} 
      \alpha_1 \textcolor{red}{1} + \ldots + \alpha_{n+1} \textcolor{red}{1} = 0 \\
      \alpha_1 \textcolor{Green}{t_1} + \ldots + \alpha_{n+1} \textcolor{Green}{t_{n+1}} = 0  \\
      \vdots \\ 
            \alpha_1 \textcolor{Blue}{t_1^n} + \ldots + \alpha_{n+1} \textcolor{Blue}{t_{n+1}^n} = 0  \\ \end{array} \right.
    \]
    
    Logo, $(\alpha_1, \alpha_2, \ldots, \alpha_{n+1})$ é solução do sistema homogêneo
    \[
    \left( \begin{array}{cccc} \textcolor{red}{1} & \textcolor{red}{1} & \textcolor{red}{\ldots} & \textcolor{red}{1} \\ \textcolor{Green}{t_1} & \textcolor{Green}{t_2} & \textcolor{Green}{\ldots} & \textcolor{Green}{t_{n+1}} \\ \vdots & \vdots & \ddots & \vdots \\ \textcolor{Blue}{t_1^n} & \textcolor{Blue}{t_2^n} & \textcolor{Blue}{\ldots} & \textcolor{Blue}{t_{n+1}^n} \end{array}\right)\left( \begin{array}{c} x_1 \\ x_2 \\ \vdots \\ x_{n+1} \end{array}\right) =  \left( \begin{array}{c} 0\\ 0 \\ \vdots \\ 0 \end{array}\right)
    \]
    
   Como $t_1, t_2, \ldots, t_{n+1}$ são diferentes, observe que a matriz obtida é uma matriz de Vandermonde. Assim, pela questão 12, temos que \[\det \left( \begin{array}{cccc} 1 & 1 & \ldots & 1 \\ t_1 & t_2 & \ldots & t_{n+1} \\ \vdots & \vdots & \ddots & \vdots \\ t_1^n & t_2^n & \ldots & t_{n+1}^n \end{array}\right) = \prod\limits_{1 \le i < j \le n+1} (t_j - t_i) \neq 0,  \] o que resulta que a única solução possível para este sistema é a trivial. Consequentemente, temos $t_1 = t_2 = \ldots = t_{n+1} =0.$ Daí, $\mathcal{B}$ é LI, e portanto uma base para $K_n[t]^{*}.$
    }

}

\exercicio{26} Seja $n > 1$ um inteiro e $I \subseteq \mathbb{R}$ um intervalo aberto. Seja $\mathcal{C}^{(n-1)}(I, \mathbb{R})$ o conjunto das funções de classe $n - 1,$ i.e. deriváveis $n - 1$ vezes com derivada $n - 1$ contínua. Dadas $f_1,\ldots, f_n \in \mathcal{C}^{(n−1)}(I, \mathbb{R}),$ o \emph{Wronskiano} de $f_1,\ldots, f_n $ é a função
\[
\fullfunction{W(f_1,\ldots, f_n)}{I}{\mathbb{R}}{t}{(W(f_1,\ldots, f_n))(t)}\]
definida como
\[
(W(f_1,\ldots, f_n))(t) = \det \left[ \begin{array}{cccc} f_1(t) & f_2(t) & \ldots & f_n(t) \\ f_1^{\prime}(t) & f_2^{\prime}(t) & \ldots & f_n^{\prime}(t) \\ \vdots & \vdots & \ddots & \vdots \\ f_1^{(n-1)}(t) & f_2^{(n-1)}(t)  & \ldots & f_n^{(n-1)}(t)  \end{array} \right]
\]
Mostre que se existir $t \in I$ tal que $(W(f_1,\ldots, f_n))(t) \neq 0$ então $\{ ff_1,\ldots, f_n \} \subset \mathcal{C}^{(n-1)}(I, \mathbb{R})$ é $\mathbb{R}$-linearmente independente.

Observe que a recíproca não é verdadeira. Por exemplo, seja $ I = (-1, 1), f_1  \colon t \to t^3, f_2 \colon t \to \abs{t^3}.$ O conjunto $\{ f_1, f_2 \}$ é $\mathbb{R}$-linearmente independente, mas $(W(f_1, f_2))(t) = 0$ para todo $t \in (-1, 1).$

\solucao{}

\exercicio{27} Seja $V$ um $K$-espaço vetorial de dimensão finita $n$ e sejam $f_1, f_2, \ldots, f_r \in V^{*}.$ Defina
\[f_1 \wedge f_2 \wedge \ldots \wedge f_r \colon V \times V \times \ldots \times V \to K\]
por $f_1 \wedge f_2 \wedge \ldots \wedge f_r (v_1, v_2, \ldots, v_r) = \det f_i(v_j).$
\dividiritens{
    \task[\pers{a}] Verifique que $f_1 \wedge f_2 \wedge \ldots \wedge f_r $ é $r$-linear e alternada.
    \task[\pers{b}] Mostre que $f_1 \wedge f_2 \wedge \ldots \wedge f_r  \neq 0$ se, e somente se $\{ f_1, f_2, \ldots, f_r\}$ é linearmente independente. 
        \task[\pers{c}] Prove que se $\{ f_1, f_2, \ldots, f_n\}$é uma base de $V^{*}$ então o conjunto
        \[
        \{f_J = f_{j_1} \wedge f_{j_2} \wedge \ldots \wedge f_{j_r} \}, \mbox{ para todo } J = \{j_1 < j_2 < \ldots j_r \} \subset \{1,2, \ldots, n \} \}
        \]
        é uma base de $\mathcal{A}_r(V).$
                \task[\pers{d}]  Sejam $B$ de uma base de $V$ e $B^{*} =  \{ f_1, f_2, \ldots,  f_n\}$ sua base dual. Descreva a base de $\mathcal{A}_r(V)$ que obtemos usando o item anterior. (A forma linear $f_1 \wedge f_2 \wedge \ldots \wedge f_r$ é chamada de \emph{produto exterior} dos funcionais $f_1, f_2, \ldots, f_r.$)
    }
    
    \solucao{}


\textbf{\textcolor{Red}{Questões Suplementares}}


\exercicio{28} Considere a matriz
\[
A = \left(\begin{array}{ccccc} \frac{1}{x_1 + y_1} & \frac{1}{x_1 + y_2} & \frac{1}{x_1 + y_3} & \ldots & \frac{1}{x_1 + y_n} \\
\frac{1}{x_2 + y_1} & \frac{1}{x_2 + y_2} & \frac{1}{x_2 + y_3} & \ldots & \frac{1}{x_2 + y_n} \\
\frac{1}{x_3 + y_1} & \frac{1}{x_3 + y_2} & \frac{1}{x_3 + y_3} & \ldots & \frac{1}{x_3 + y_n} \\
\vdots & \vdots & \vdots & \ddots & \vdots \\
\frac{1}{x_n + y_1} & \frac{1}{x_n + y_2} & \frac{1}{x_n + y_3} & \ldots & \frac{1}{x_n + y_n} \\
\end{array} \right),
\]
onde $x_i + y_j \neq 0$ para $1 \le i,j \le n.$ Mostre que o determinante dessa matriz, conhecido por \emph{determinante de Cauchy}, é dado por
\[
\det A = \frac{\prod\limits_{i > j}^n (x_i - x_j)(y_i - y_j)}{\prod\limits_{i,j = 1}^n (x_i + y_j)}
\]
%https://books.google.com.br/books?id=CSDbVU1Eg3UC&pg=PA59&lpg=PA59&dq=%3Ddet(AD+-+BD%5E%7B-1%7DCD)&source=bl&ots=lBFjYGdw_d&sig=ACfU3U1E0lDFdZ5q_sA832oS704PaHrcTQ&hl=pt-BR&sa=X&ved=2ahUKEwi5wdrOpovkAhWNLLkGHQd9AwQQ6AEwCnoECAgQAQ#v=onepage&q&f=false -ex26 pg 61
\solucao{}

\exercicio{29} O determinante da \emph{matriz circulante} $n \times n$ é dado por
\[
\det \begin{bmatrix} a_1 & a_2 & a_3 & \ldots & a_n \\
a_n & a_1 & a_2 & \ldots & a_{n-1} \\
\vdots & \vdots & \vdots & \ddots & \vdots \\
a_3 & a_4 & a_5 & \ldots & a_2 \\
a_2 & a_3 & a_4 & \ldots & a_1
\end{bmatrix} = (-1)^{n-1} \prod\limits_{j = 0}^{n-1} \left( \sum\limits_{k = 1}^n \zeta^{jk} a_k \right),
\]
onde $\zeta = e^{\frac{2 \pi i}{n}}.$ Encontre o determinante da matriz circulante $n \times n$ dada por
\[
A = \begin{bmatrix} 1 & 4 & 9 & \ldots & n^2  \\
n^2 & 1 & 4 & \ldots & (n-1)^2 \\
\vdots & \vdots & \vdots & \ddots & \vdots \\
9 & 16 & 25 & \ldots & 4 \\
4& 9 & 16 & \ldots & 1
\end{bmatrix}.
\]
%https://books.google.com.br/books?id=CSDbVU1Eg3UC&pg=PA59&lpg=PA59&dq=%3Ddet(AD+-+BD%5E%7B-1%7DCD)&source=bl&ots=lBFjYGdw_d&sig=ACfU3U1E0lDFdZ5q_sA832oS704PaHrcTQ&hl=pt-BR&sa=X&ved=2ahUKEwi5wdrOpovkAhWNLLkGHQd9AwQQ6AEwCnoECAgQAQ#v=onepage&q&f=false - ex29 - pg63

%Problems and Solutions in Introductory and Advanced Matrix Calculus Por W.-H. Steeb, Willi-Hans Steeb

%https://books.google.com.br/books?id=CSDbVU1Eg3UC&pg=PA59&lpg=PA59&dq=%3Ddet(AD+-+BD%5E%7B-1%7DCD)&source=bl&ots=lBFjYGdw_d&sig=ACfU3U1E0lDFdZ5q_sA832oS704PaHrcTQ&hl=pt-BR&sa=X&ved=2ahUKEwi5wdrOpovkAhWNLLkGHQd9AwQQ6AEwCnoECAgQAQ#v=onepage&q&f=false problema 8 - pg50 Csansky algorithm eingevector and eigenvalue

\solucao{}

\exercicio{30} Sejam $A, B \in \mathcal{M}_n(K)$ duas matrizes invertíveis, tais que
\[
A^{-1} + B^{-1} = (A + B)^{-1}
\]
\dividiritens{
    \task[\pers{a}] Se $K = \mathbb{R},$ mostre que $\det A = \det B.$
    \task[\pers{b}] Se $K = \mathbb{C},$ mostre que pode ocorrer $\det A \neq \det B,$ mas é válido que $\abs{det A} = \abs{\det B}.$
    }
    %https://math.stackexchange.com/questions/3062983/let-a-b-be-n-times-n-with-n-ge-2-nonsingular-matrices-with-real-entries-s
    \solucao{}
\end{document} 